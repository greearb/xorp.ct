% -*- mode: latex; tex-main-file: "pospaper.tex" -*-

\section{Summary}

%% We believe that much good Internet research is being frustrated by an
%% inability to deploy experimental router software at points in the
%% network where it makes most sense.  These problems affect a wide range
%% of research, including routing protocols themselves, active queue
%% management schemes, and so-called ``middlebox'' functionality such as
%% our own traffic normalizer~\cite{norm}.  Many of these problems would
%% not exist if the router software market more closely resembled the
%% end-system software market, which has well defined APIs for
%% application software, and high performance reliable open-source
%% operating systems that permit kernel protocol experimentation.  Our
%% vision for XORP is to provide just such an open
%% platform; one that is stable and fully featured enough for serious
%% production use (initially on edge-routers), but designed from the very
%% outset to support extensibility and experimentation without
%% compromising the stability of the core platform.

%% Thus, while XORP is primarily intended to {\em enable} research, we
%% also believe that it will also further knowledge about how to
%% construct robust extensible networking systems.

%% Finally, while we do not expect to change the whole way the router
%% software market functions, it is not impossible that the widespread
%% use of an open software platform for routers might have this effect.
%% The ultimate measure of our success would be if commercial router
%% vendors either adopted XORP directly, or opened up their software
%% platforms in such a way that a market for router application software
%% is enabled.

We believe that innovation in the core protocols supporting the
Internet is being seriously inhibited by the nature of the router
software market.  Furthermore, little long term research is being
done, in part because researchers perceive insurmountable obstacles to
experimentation and deployment of their ideas.

In an attempt to change the router software landscape, we have built
an extensible open router software platform.  The next few years will
be critical.  We now have a stable core base running, consisting of
around 500,000 lines of C++.  In the next phase we need to involve the
academic community, both as early adopters, and to flesh out the long
list of desirable functionality that we do not yet support.  If we are
successful, XORP will become a true {\it production-quality} platform.
We already have interest from a number of networking startup
companies.  The road ahead will not be easy, but we believe that some
significant change to permit innovation is necessary, and the
consequences of lack of innovation in the long run will be serious.


